\newpage

\setcounter{page}{1}
\makecvfooter
  {Jerald Thomas}
  {Teaching Statement}
  {Page \thepage \hspace{1pt} of \pageref{teaching_last}}


\makecvheader[C]
\doublespacing

\cvsection{Teaching Statement}

My primary goal as an educator is to graduate students who will be able to contribute to the field of computer science and technology. This requires that they can learn independently, solve complex problems, and adapt to changing technologies. To accomplish this goal, my teaching philosophy is broadly composed of four principles. Firstly, I will provide a firm grasp of fundamental elements, giving the student a solid base to build upon as they progress in their career. Next, I will guide students to develop and hone their critical thinking skills to determine the best way to approach a problem. Thirdly, I will make sure that my teaching is adaptable, both to the individual student and to the class as a whole. Finally, I will strive to create a safe and open environment, ensuring all students get the most out of their educational experience.

The feedback I have received from students, TAs, and other instructors shows appreciation for the level of attention and care I put into teaching. From course design and preparation to working with students one-on-one, I am devoted to ensuring that each student gets the most out of my courses. Though I have only been the instructor of record for two courses so far, my feedback has been overwhelmingly positive. This is something that I have taken great pride in and is a standard that I will strive to maintain. A Google Drive folder containing feedback and course materials from my previous courses can be found using this URL link: \url{https://bit.ly/3GpISxe}.

\subsection*{Focusing on Fundamental Elements}
\vspace{-0.5cm}
The objective of any course should be a possession of the fundamental concepts and the ability to use these concepts to continue learning the subject after the course has ended. For example, it is less important whether a student can memorize the API of a given platform and more so whether the student, given the API documentation, can internalize and understand the information in a way that leads to effective programming and use of that platform. In computer science, where some knowledge gained in class can become obsolete shortly after graduating, the ability to learn on one's own and keep up with the cutting edge of technology is a crucial component to be hireable and maintain a stable and healthy career. This is achieved best through a strong grasp of the fundamental elements.

\subsection*{Developing Critical Thinking Skills}
\vspace{-0.5cm}
The best solution to a problem is rarely apparent in computer science. Students must learn how to effectively analyze and dissect problems into manageable pieces to find optimal solutions. While it is necessary to continually develop critical thinking as a skill, it is the role of the educator to instill the necessary foundation. To accomplish this, programming projects and other more extensive assignments will make up the majority of my courses’ requirements. Unlike exams and traditional homework exercises, projects allow me as an instructor to present complex and multi-faceted problems for students to analyze, break down, and solve. The best way to improve critical thinking skills is repeated exposure and practice, and a course structure where most of the grade comes from projects is an optimal way to achieve this.

\subsection*{Adapting My Teaching Style}
\vspace{-0.5cm}
It is critial as a teacher to understand that all students learn differently. Additionally, students come from many different backgrounds, some with more hurdles to overcome than others. Because of this, I believe that instructors must exercise a student-driven, multi-modal, and individualized approach whenever possible. As both an instructor and a teaching assistant, I have observed the need to teach to the individual student. I had the most success teaching by altering my approach on a per-student basis and incorporating as many different approaches as possible in lectures and course materials. My students have noted special appreciation for the effort I put into getting to know each student individually and my ability to adapt my teaching style to best match their learning style. However, I understand that I will inevitably encounter a scenario where I struggle to adapt to a particular student's needs throughout my teaching career. In those situations, I feel it is imperative to seek guidance from more experienced professors, university resources, or the student themselves to figure out how to meet their needs best.

\subsection*{Creating a Safe and Open Environment}
\vspace{-0.5cm}
Students have to feel welcome, safe, and respected to be successful in any course, and it is my responsibility to ensure that this happens in all of the courses that I teach. Fostering this culture begins with getting to know my students and allowing my students to get to know me. Educators are more successful in facilitating a safe and open learning environment when they understand where their students are coming from. Crucially, students in my courses will be treated with respect and expected to do the same with myself and their peers; this will be made clear in both the syllabus, in-person on the first day of classes, and addressed as needed throughout the semester. I will also make sure that students are aware of the resources offered by the university if they do not feel comfortable approaching the department or myself.

Some specific examples of how I intend to create an equitable and welcoming classroom are: 
\begin{itemize}
    \item Maintain an open-door policy to promote trust and openness between the students and myself
    \item Invite high performing students from underrepresented backgrounds to TA for the course in future semesters
    \item Implement blind grading where possible to remove implicit bias
    \item Use inclusive language and put extra effort into making sure I am pronouncing students' names correctly
    \item Actively ensure that I do not interact with one demographic of students more than any other
\end{itemize}

There are several other ways to ensure that the classroom is an inclusive and equitable place for my students to learn. I would enjoy speaking with you about other ways to further this objective and garner any input and ideas you have on the matter.

\section*{Future Classes}
\vspace{-0.5cm}
I have spent the past six years pursuing my Ph.D. to obtain a faculty position, and research has been my focus for the majority of that time. However, one of the main reasons I decided on this career path is that I love teaching, and I am excited at the prospect of beginning my teaching career. I am confident in teaching fundamental computer science courses in addition to higher-level courses on immersive technologies, human-computer interaction, and graphics. Developing and teaching human-computer interaction courses in particular is a way that I believe I could build upon the existing courses offered at UWM.

As my undergraduate degree was in electrical and computer engineering, I am particularly keen on teaching topics closer to the hardware, such as operating systems and computer architecture. Additionally, I have recently started using the Rust programming language and would love to build a course teaching systems-level concepts using Rust as it represents an important paradigm shift in computer programming.

Another interest of mine is to create a course built around immersive technologies that focuses on interdisciplinary collaboration. The idea is inspired by a course developed by Dr. Benjamin Lok at the University of Florida titled ``VR for the Social Good.’’ The goal would be for students not necessarily majoring in computer science to learn how to use virtual and augmented reality as a platform and apply it to a specific problem within their discipline; the focus would be more on virtual reality application rather than theory. Computer science is inherently an interdisciplinary field, and a course like this would highlight that fact.

\label{teaching_last}
