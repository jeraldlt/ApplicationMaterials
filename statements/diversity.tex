\newpage

\setcounter{page}{1}
\makecvfooter
  {Jerald Thomas}
  {Diversity Statement}
  {Page \thepage \hspace{1pt} of \pageref{diversity_last}}

\makecvheader[C]
\doublespacing

\cvsection{Diversity Statement}

A community with a diverse population is the most ideally situated to thrive and produce the best people and ideas. Unfortunately, our discipline is severely lacking when it comes to diversity as a whole. A glaring example is that women only make up 18\% of students pursuing computer science bachelor's degrees in the United States. A lack of diversity not only harms individuals within minority populations but also harms our field, science as a whole, and our society. It is disheartening to imagine the number of people who could have been great scientists and engineers but could not achieve this due to discrimination, and I am committed to doing my part in making sure that our field becomes more diverse, equitable, and inclusive.

A cornerstone of my approach to teaching and mentorship is the understanding that not all students have been afforded the same cultural and socioeconomic benefits as myself or other students. I am not blind to the fact that I currently have, and have had, many privileges and affordances provided to me. Because the system I have benefited from is so ingrained in our culture, and because I tend to be on the receiving end, it can be challenging to identify when I benefit from or even advance inequality. This lack of sight is a shortcoming that I have been working on improving, and I will continue to do so. The first steps I have taken to improve are consistently reflecting upon my actions, obtaining critique from my students and peers, and seeking additional resources.

However, passive improvement is not enough, and I firmly believe that I must take intentional actions when I interact with students and during my other roles as a faculty member. The position of faculty member and the power and responsibilities that come with it will be new to me. Nevertheless, in the past, I have used my influence where I could to empower underrepresented individuals. For example, I mentored FIRST robotics teams for six years, and I believe my most significant contribution was cultivating a female dominant programming team. I also volunteered several summers at engineering camps for junior-high girls. In addition, as a senior Ph.D. student, I have worked with my advisor to create a diverse and welcoming lab environment, where half of the students are female, and most students are from minority backgrounds.

As an instructor, I made it explicitly clear at the beginning of the semester and as needed that harassment and prejudice have no place in my classroom nor the university at large. I also put extra effort into removing any implicit bias towards students by implementing blind grading whenever possible, and I kept track of the students that I called on and otherwise interacted with during class sessions to ensure I did not show any preferential treatment. This is noticed by the students and even mentioned in instructor reviews. There are several great resources that I have found since teaching my last course that have given me even more ideas on how to produce a diverse and equitable classroom environment.

In the future, I will participate in organizing and supporting diversity and affirmative action activities, such as taking part in organizing the Computing CARES program. In addition, creating and maintaining a diverse research group will be one of my top priorities. I understand that I do not come near to knowing all of the best ways to support the department’s goals of diversity, inclusivity, equality, and justice; I am currently, and will continue to be, open to any suggestions you may have for how I may bolster these goals held by the department and myself.


\label{diversity_last}
