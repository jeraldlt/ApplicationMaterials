\newpage

\setcounter{page}{1}
\makecvfooter
  {Jerald Thomas}
  {Diversity Statement}
  {Page \thepage \hspace{1pt} of \pageref{diversity_last}}

\makecvheader[C]
%\doublespacing

\cvsection{Diversity Statement}

A community with a diverse population is the most ideally situated to thrive and produce the best people and ideas. Unfortunately, our discipline is severely lacking when it comes to diversity as a whole. A glaring example is that women only make up 18\% of students pursuing computer science bachelor's degrees in the United States. A lack of diversity not only harms individuals within minority populations but also harms our field, science as a whole, and our society. It is disheartening to imagine the number of people who could have been great scientists and engineers but could not achieve this due to discrimination, and I am committed to doing my part in making sure that our field becomes more diverse, equitable, and inclusive.

A cornerstone of my approach to teaching and mentorship is the understanding that not all students have been afforded the same cultural and socioeconomic benefits as myself or other students. I acknowledge that I have many privileges and affordances provided to me; because the system I have benefited from is so ingrained in our culture, and because I tend to be on the receiving end, it can be challenging to identify when I benefit from or even advance inequality. This lack of sight is a shortcoming that I have been working on improving, and I am comitted to continuing self-improvement in this area. The first steps I have taken to improve are consistently reflecting upon my interactions, obtaining critique from my students and peers, and seeking additional resources.

However, passive improvement is not enough, and I firmly believe that I must take intentional actions when I interact with students and during my other roles as a faculty member. I have tried to use my influence where I could to empower underrepresented individuals in the past. For example, I mentored FIRST robotics teams for six years where I was the lead mentor for the programming subteam. I intentially encouraged women to take an active part in the programming, and when I left the programming team had a female lead and was mostly comprised of women. When I was an undergraduate I volunteered in the planning and execution of engineering camps for junior high girls for several summers. In addition, as student volunteer chair for the 2019 IEEE Conference on Virtual Reality and 3D User Interfaces I put special effort towards creating a diverse and inclusive cohort of student volunteers. 

As an instructor, I made it explicitly clear at the beginning of the semester and as needed that harassment and prejudice have no place in my classroom nor the university at large. I also put extra effort into removing any implicit bias towards students by implementing blind grading whenever possible, and I kept track of the students that I called on and otherwise interacted with during class sessions to ensure I did not show any preferential treatment. However, I recognize that I was not always successful at this. Fortunately, I have had TAs and students that felt comfortable enough to inform me when they thought I had not treated a TA or student particularly equitable. I understand that I will not always be perfect, but I will always strive to create an environment where my students and TAs feel able to approach me with concerns.

In the future, I will participate in organizing and supporting diversity and affirmative action activities, such as taking part in women in computing initiatives. In addition, creating and maintaining a diverse research group will be one of my top priorities. Finally, I understand that I do not know all of the best ways to support diversity, inclusivity, equality, and justice; I am currently, and will continue to be open to any suggestions that you, other faculty members, or my students have for how I may bolster these goals held by the department and myself.

\label{diversity_last}
